\clearpage                                  % В том числе гарантирует, что список литературы в оглавлении будет с правильным номером страницы
%\hypersetup{ urlcolor=black }               % Ссылки делаем чёрными
%\providecommand*{\BibDash}{}                % В стилях ugost2008 отключаем использование тире как разделителя
\urlstyle{rm}                               % ссылки URL обычным шрифтом
\ifdefmacro{\microtypesetup}{\microtypesetup{protrusion=false}}{} % не рекомендуется применять пакет микротипографики к автоматически генерируемому списку литературы

% Режим с подсписками
%\insertbiblioexternal                      % Подключаем Bib-базы: статьи, не являющиеся статьями автора по теме диссертации
% Для вывода выберите и расскомментируйте одно из двух
%\insertbiblioauthor                        % Подключаем Bib-базы: работы автора единым списком
%\insertbiblioauthorgrouped                 % Подключаем Bib-базы: работы автора сгруппированные (ВАК, WoS, Scopus и т.д.)
\ifdefmacro{\microtypesetup}{\microtypesetup{protrusion=true}}{}
\urlstyle{tt}                               % возвращаем установки шрифта ссылок URL
%\hypersetup{ urlcolor={urlcolor} }          % Восстанавливаем цвет ссылок

\chapter*{Список литературы}
\addcontentsline{toc}{Список литературы}  % Добавляем его в оглавление


\begin{thebibliography}{7}
	\bibitem{Lopatin} \textit{Лопатин\ А.\ С.} (2005) Стохастическая оптимизация
в информатике. Метод отжига. // Сайт Math.spbu.ru. URL: https://www.math.spbu.ru/user/gran/\\optstoch.htm (дата обращения: 20.01.2020)
	\bibitem{Shamin} \textit{Шамин\ Р.\ В.} (2019) Практическое руководство по машинному обучению. // Москва: Научный канал.
	\bibitem{Weise} \textit{Weise\ T.} (2009) Global Optimization Algorithms --- Theory and
Application. // Self-Published.
	\bibitem{Pedersen} \textit{Pedersen\ M.\ E.\ H.} (2010) Tuning and Simplifying Heuristical Optimization. Thesis for the degree of Doctor of Philosophy. // University of Southampton.
	\bibitem{Nedjah} \textit{Nedjah\ N.} and \textit{Mourelle\ L.} (2006) Swarm Intelligent Systems. // Berlin, Heidelberg: Springer Berlin Heidelberg.
\end{thebibliography}
