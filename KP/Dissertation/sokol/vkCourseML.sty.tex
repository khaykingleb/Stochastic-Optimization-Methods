\usepackage[utf8]{inputenc}
%\usepackage[sort]{cite}
\usepackage[colorlinks,urlcolor=blue]{hyperref}
\usepackage{amssymb}
\usepackage{amsmath}
\usepackage[russian]{babel}
\usepackage{array}
\usepackage{theorem}
\usepackage{ifthen}
\usepackage[ruled,section]{algorithm}
\usepackage[noend]{algorithmic}
\usepackage[all]{xy}
\usepackage{graphicx}
\usepackage{floatflt}
%%\usepackage{pb-diagram}
\usepackage{multicol}
\usepackage[footnotesize]{caption2}
\usepackage{mathrsfs}
\usepackage{color}
\usepackage{verbatim}

\title{Математические методы обучения по прецедентам}
\author{К. В. Воронцов}

\def\vkWarning{\par{\footnotesize\rm
		Материал находится в~стадии разработки, может содержать ошибки и~неточности.
		Автор будет благодарен за~любые замечания и~предложения,
		направленные по~адресу {\tt vokov@forecsys.ru},
		либо высказанные в~обсуждении страницы
		<<Машинное обучение (курс лекций, К.В.Воронцов)>>
		вики-ресурса {\tt www.MachineLearning.ru}.
		
		Перепечатка фрагментов данного материала без согласия автора является плагиатом.
}}

\def\vkMakeTitle{%
	\maketitle\thispagestyle{empty}%
	\vkWarning
	\TableOfContents
	\newpage
}

\textheight=240mm
\textwidth=160mm
\topmargin=-14mm
\headsep=7mm
%\oddsidemargin=7mm\evensidemargin=-3mm
\oddsidemargin=0mm\evensidemargin=0mm
\marginparwidth=36pt
\tolerance=3000
\hbadness=2000
%\flushbottom
\raggedbottom
% подавить эффект "висячих стpок"
\clubpenalty=10000
\widowpenalty=10000
\renewcommand{\baselinestretch}{1}
%\renewcommand{\baselinestretch}{1.1} %для печати с большим интервалом

% Очень стРРААШШные команды для вставки самостоятельных документов целиком
\newenvironment{inputs}{
	\renewcommand{\documentclass}[2][0]{\relax}
	\renewcommand{\usepackage}[2][0]{\relax}
	\renewcommand{\vkMakeTitle}{\relax}
	%\renewcommand{\thispagestyle}[1]{\relax}
	%\renewcommand{\newpage}{\relax}
	%\renewcommand{\TableOfContents}{\relax}
	\renewcommand{\bibliographystyle}[1]{\relax}
	\renewcommand{\bibliography}[1]{\relax}
	\renewenvironment{document}{\relax}{\relax}
}{}

% Комментарии
\newcommand{\vkRem}[1]{}
\newcounter{vkTodoCount}
\def\thevkTodoCount{\arabic{vkTodoCount}}
\newcommand{\TODO}[1]{
	\par\medskip{\small\color{red}%
		\def\subsection##1{\par\smallskip\noindent{\normalfont\large\sffamily\bfseries ##1}\par}%
		\def\subsubsection##1{\par\smallskip\noindent{\normalfont\normalsize\sffamily\bfseries ##1}\par}%
		\def\paragraph##1{\par\smallskip\noindent{\normalfont\normalsize\sffamily\bfseries ##1}}%
		\marginpar{\raisebox{-3ex}{\color{red}ToDo$^{\refstepcounter{vkTodoCount}\thevkTodoCount}$}}%
		#1}\par}
%\newcommand{\TODO}[1]{\par\bigskip\noindent\parbox{16cm}{\sf TODO:\par\small #1}\bigskip\par}
%\renewcommand{\TODO}[1]{}

% Расширенная и урезанная версии
\newenvironment{Full}{}{}
\newenvironment{Short}{}{}
\definecolor{FullVerColor}{rgb}{0,0.6,0}
\definecolor{ShortVerColor}{rgb}{0.7,0,0.7}
\newcommand{\MakeFullDraft}{%
	\gdef\FullVer##1{{\color{FullVerColor}##1}}%
	\gdef\ShortVer##1{{\color{ShortVerColor}##1}}%
	\renewenvironment{Full}{\color{FullVerColor}}{}
	\renewenvironment{Short}{\color{ShortVerColor}}{}
}
\newcommand{\MakeFullVer}{%
	\gdef\FullVer##1{##1}%
	\gdef\ShortVer##1{\relax}%
	\renewenvironment{Full}{}{}
	\renewenvironment{Short}{\comment}{\endcomment}
}
\newcommand{\MakeShortVer}{%
	\gdef\FullVer##1{\relax}%
	\gdef\ShortVer##1{##1}%
	\renewenvironment{Full}{\comment}{\endcomment}
	\renewenvironment{Short}{}{}
}
%\MakeFullDraft
\MakeFullVer

% Библиография
\definecolor{BibColor}{rgb}{0,0,1}
\def\BibUrl#1{\\{\def~{\char126}\footnotesize\color{BibColor}\texttt{http://#1}}}

% Мои команды для сокращения текста и формул
\renewcommand{\em}{\it}
\newcommand{\lastTerm}{}
\newcommand{\vkTerm}[1]{{\em #1}\renewcommand{\lastTerm}{#1}}
\newcommand{\vkTrans}[2][void]{%
	\ifthenelse{\equal{#1}{void}}%
	{\footnote{\lastTerm --- #2}}%
	{\footnote{#1 --- #2}}%
}
\renewcommand{\vkTrans}[2][void]{ (#2)}
%\newcommand{\vkAddTrans}[2]{{#1}}
%\newcommand{\vkAddTrans}[2]{{\sl #1 \small(#2)}}
%\newcommand{\vkAddTrans}[2]{{\sl #1}\footnote{#2.}}

% Полезные команды для математики
\newcommand{\const}{\mathrm{const}}
\newcommand{\rang}{\mathop{\mathrm{rk}}}
\newcommand{\Tr}{\mathop{\mathrm{tr}}}
%\newcommand{\Tr}{\mathsf{Tr\,}}
\newcommand{\tsum}{\mathop{\textstyle\sum}\limits}
\newcommand{\tprod}{\mathop{\textstyle\prod}\limits}
\newcommand{\tvee}{\mathop{\textstyle\bigvee}\limits}
\newcommand{\twedge}{\mathop{\textstyle\bigwedge}\limits}
\newcommand{\diag}{\mathop{\mathrm{diag}}}
\newcommand{\argmin}{\mathop{\rm arg\,min}\limits}
\newcommand{\argmax}{\mathop{\rm arg\,max}\limits}
\newcommand{\sign}{\mathop{\rm sign}\limits}
\newcommand{\med}{\mathop{\rm med}\limits}
\newcommand{\SoftMax}{\mathop{\rm SoftMax}\nolimits}
\newcommand{\Dir}{\mathop{\rm Dir}\nolimits}

\renewcommand{\geq}{\geqslant}
\renewcommand{\leq}{\leqslant}
\newcommand{\ophi}{\phi}
\renewcommand{\phi}{\varphi}
\newcommand{\eps}{\varepsilon}
\newcommand{\kapa}{\varkappa}
\newcommand{\emset}{\varnothing}
\newcommand{\cond}{\mspace{3mu}{|}\mspace{3mu}}
\newcommand{\refuse}{\o}
\newcommand{\Loss}{\mathscr{L}}
\newcommand{\Expect}{\mathsf{E}}
\newcommand{\Disp}{\mathsf{D}}
\newcommand{\Var}{\mathsf{D}}

\def\RR{\mathbb{R}}
\def\DD{\mathbb{D}}
\def\cL{\mathscr{L}}
\def\cF{\mathscr{F}}
\def\cG{\mathscr{G}}
\def\cJ{\mathcal{J}}
\def\cN{\mathcal{N}}
\def\cB{\mathscr{B}}
\def\cK{\mathscr{K}}
\def\fF{\mathfrak{F}}
\def\fI{\mathfrak{I}}
\def\fM{\mathfrak{M}}

\newcommand{\what}{\widehat}
\newcommand{\wtil}{\widetilde}
\newcommand{\ttil}{\raisebox{0.2ex}{\tiny$\sim$}}
%\newcommand{\T}{\mathsf{T}}
\newcommand{\T}{\textsf{\upshape т}}
\newcommand{\phanT}{^{\vphantom{\T}}}
\newcommand{\ExpecT}{\Expect^{\phanT}}

\newcommand{\scal}[2]{\left\langle #1,#2 \right\rangle}
%\newcommand{\scal}[2]{{#1}^\T{#2}}
%\newcommand{\Near}[3]{{#1}_{#3,#2}}
\newcommand{\Near}[3]{{#1}^{(#3)}_{#2}}
%\newcommand{\Near}[3]{{#1}_{#3:#2}}

%\newcommand{\vkAnd}{\mathop{\&}}
\newcommand{\vkAnd}{\wedge}
\newcommand{\vkQual}{Q}
\newcommand{\vkAtop}[2]{\genfrac{}{}{0pt}{1}{#1}{#2}}

% Гипергеометрическое распределение
%\newcommand{\HyperGeom}[4]{\Gamma_{#1,#2}^{#3,#4}}
%\newcommand{\HyperGeom}[4]{\Gamma_{#1}^{3}{}_{#2}^{#4}}
%\newcommand{\HyperGeom}[4]{\Gamma\!
%    \left(\!\begin{smallmatrix}
%        #3\hfill \!& #4\hfill\\
%        #1\hfill \!& #2\hfill
%    \end{smallmatrix}\!\right)
%}

%\newcommand{\@hyper@geom}[5]{#1\!
%    \left(\!\begin{smallmatrix}
%        #4\hfill \!& #5\hfill\\
%        #2\hfill \!& #3\hfill
%    \end{smallmatrix}\!\right)
%}
%\newcommand{\HyperGeom}[4]{\@hyper@geom{h}{#1}{#2}{#3}{#4}}
%\newcommand{\HyperGeomSum}[4]{\@hyper@geom{H}{#1}{#2}{#3}{#4}}
%\newcommand{\HyperGeomRight}[4]{\@hyper@geom{H'}{#1}{#2}{#3}{#4}}

\newcommand{\@hyper@geom}[5]{{#1}_{#2}^{#4,#3}\left(#5\right)}
\newcommand{\hyper}[4]{\@hyper@geom{h}{#1}{#2}{#3}{#4}}
\newcommand{\Hyper}[4]{\@hyper@geom{H}{#1}{#2}{#3}{#4}}
\newcommand{\HyperR}[4]{\@hyper@geom{\bar{H}}{#1}{#2}{#3}{#4}}

% Абзацный отступ и пропорциональные ему отступы
\parindent=29pt
\newlength{\myparind}
\myparind=\parindent
% Примочка к fleqn.sty
%\mathindent = 2\myparind %\leftmargini

% Переопределение колонтитулов
%\renewcommand{\@oddfoot}{}
%\renewcommand{\@oddhead}{\parbox{\textwidth}{\footnotesize Воронцов~К.\,В.\hfill\thepage\\[-2ex]\hrule}}
%\renewcommand{\@evenfoot}{}
%\renewcommand{\@evenhead}{\parbox{\textwidth}{\footnotesize\thepage\hfill Математические методы обучения по~прецедентам\\[-2ex]\hrule}}
\renewcommand{\theenumii}{\asbuk{enumii}}

% Переопределение колонтитулов
\renewcommand{\ps@headings}{%
	\renewcommand{\@oddfoot}{}%
	\renewcommand{\@oddhead}{\parbox{\textwidth}{\footnotesize\rightmark\hfill\thepage\\[-2ex]\hrule}}%
	\renewcommand{\@evenfoot}{}%
	\renewcommand{\@evenhead}{\parbox{\textwidth}{\footnotesize\thepage\hfill
			К.\,В.\,Воронцов.~Вычислительные методы обучения по~прецедентам\\[-2ex]\hrule}}
}
\ps@headings

% Нумерацию формул переподчиним разделам
\@addtoreset{equation}{section}
\def\theequation{\thesection.\arabic{equation}}

% Стили параграфов
\definecolor{SecColor}{rgb}{0,0,0.5}
\newcommand{\vkSecColor}{\color{SecColor}}
\renewcommand\section{\@startsection{section}{1}{\z@}%
	{3.5ex \@plus 1ex \@minus .2ex}%
	{2.3ex \@plus.2ex}%
	{\normalfont\Large\sffamily\bfseries\vkSecColor}}
\renewcommand\subsection{\@startsection{subsection}{2}{\z@}%
	{3.25ex\@plus 1ex \@minus .2ex}%
	{1.5ex \@plus .2ex}%
	{\normalfont\large\sffamily\bfseries\vkSecColor}}
\renewcommand\subsubsection{\@startsection{subsubsection}{3}{\z@}%
	{3.25ex\@plus 1ex \@minus .2ex}%
	{1.5ex \@plus .2ex}%
	{\normalfont\normalsize\sffamily\bfseries\vkSecColor}}
\renewcommand\paragraph{\@startsection{paragraph}{4}{\z@}%
	{3.25ex\@plus1ex\@minus.2ex}%
	{-1ex\@plus-.5ex}%
	{\normalfont\normalsize\sffamily\bfseries\vkSecColor}}
\renewcommand\subparagraph{\@startsection{subparagraph}{5}{\parindent}%
	{3.25ex \@plus1ex \@minus.2ex}%
	{-1ex\@plus-.5ex}%
	{\normalfont\normalsize\sffamily\bfseries\vkSecColor}}

% Значок параграфа \S только в номерах разделов уровня subsection
\renewcommand\thesubsection{\S\arabic{section}.\arabic{subsection}}
\renewcommand\thesubsubsection{\arabic{section}.\arabic{subsection}.\arabic{subsubsection}}

% Отменить значок параграфа \S в заголовке оглавления
\newcommand{\TableOfContents}{%
	\let\vkTempS=\S
	\def\S{}
	\tableofcontents
	\let\S=\vkTempS
}

% Оформление окружений типа теорем
%\setlength{\theorempreskipamount}{0pt}
%\setlength{\theorempostskipamount}{0pt}
\newcounter{vkCorollaryCount}
\def\thevkCorollaryCount{\arabic{vkCorollaryCount}}
\theoremstyle{plain}
% Шаманские притопывания, чтобы ставить точку после номера теоремы
\gdef\th@plain{\normalfont
	\def\@begintheorem##1##2{%
		\item[\hskip\labelsep\theorem@headerfont ##1\ ##2. ]\setcounter{vkCorollaryCount}{0}}%
	\def\@opargbegintheorem##1##2##3{%
		\item[\hskip\labelsep\theorem@headerfont ##1\ ##2 (##3). ]\setcounter{vkCorollaryCount}{0}}%
}
% Теоремы наклонным шрифтом
\theorembodyfont{\rmfamily\slshape}
\newtheorem{vkTheorem}{Теорема}[section]
\newtheorem{vkLemma}[vkTheorem]{Лемма}
\newtheorem{vkState}[vkTheorem]{Утв.}
\newtheorem{vkDef}{Опр.}[section]
\newtheorem{vkHypothesis}{Гипотеза}[section]
\newtheorem{vkProblem}{Задача}[section]
% Теоремы прямым шрифтом
\theorembodyfont{\rmfamily}
\newtheorem{vkExample}{Пример}[section]
\newtheorem{vkHeuristic}{Эвристика}
\newtheorem{vkRemark}{Замечание}[section]
\newtheorem{vkState-rm}{Утв.}
% Теоремы мелким прямым шрифтом (упражнения в конце главы)
\theorembodyfont{\rmfamily\footnotesize}
\newtheorem{vkExerc}{Упр.}[section]
\newtheorem{vkWork}{Задание}[section]
% Теоремо-подобные штучки
\newenvironment{vkSolution}[1]%
{\begingroup\footnotesize\par\noindent{\bf Решение~\ref{#1}.}}%
{\par\medskip\endgroup}
\newenvironment{vkProof}%
{\par\noindent{\bf Доказательство.\par\nopagebreak}}%
{\hfill$\scriptstyle\blacksquare$\par\medskip}
\newenvironment{vkCorollary}%
{\par\bigskip\noindent{\bf Следствие\ \refstepcounter{vkCorollaryCount}\thevkCorollaryCount. }}%
{\par\medskip}
\newenvironment{vkFigure}%
{\begin{center}}%
	{\end{center}}
% Контрольные вопросы
\newcommand{\vkQuestion}[1]{%
	\begin{vkQuest}#1\end{vkQuest}%
}
\newcommand{\vkAnswer}[1]{#1}
\renewcommand{\vkQuestion}[1]{}


% Оформление алгоритмов в пакетах algorithm, algorithmic
%\definecolor{KwColor}{rgb}{0,0,0.9}
\newcommand{\vkKw}[1]{{\bf\color{SecColor} #1}}
\renewcommand{\algorithmicrequire}{\rule{0pt}{2.5ex}\vkKw{Вход:}}
\renewcommand{\algorithmicensure}{\vkKw{Выход:}}
\renewcommand{\algorithmicend}{\vkKw{конец}}
\renewcommand{\algorithmicif}{\vkKw{если}}
\renewcommand{\algorithmicthen}{\vkKw{то}}
\renewcommand{\algorithmicelse}{\vkKw{иначе}}
\renewcommand{\algorithmicelsif}{\algorithmicelse\ \algorithmicif}
\renewcommand{\algorithmicendif}{\algorithmicend\ \algorithmicif}
\renewcommand{\algorithmicfor}{\vkKw{для}}
\renewcommand{\algorithmicforall}{\vkKw{для всех}}
\renewcommand{\algorithmicdo}{}
\renewcommand{\algorithmicendfor}{\algorithmicend\ \algorithmicfor}
\renewcommand{\algorithmicwhile}{\vkKw{пока}}
\renewcommand{\algorithmicendwhile}{\algorithmicend\ \algorithmicwhile}
\renewcommand{\algorithmicloop}{\vkKw{цикл}}
\renewcommand{\algorithmicendloop}{\algorithmicend\ \algorithmicloop}
\renewcommand{\algorithmicrepeat}{\vkKw{повторять}}
\renewcommand{\algorithmicuntil}{\vkKw{пока}}
%\renewcommand{\algorithmiccomment}[1]{{\footnotesize // #1}}
\renewcommand{\algorithmiccomment}[1]{{\quad--- #1}}
\renewcommand{\listalgorithmname}{Список алгоритмов}
% Мои дополнительные команды для описания алгоритмов
\newcommand{\BEGIN}{\\[1ex]\hrule\vskip 1ex}
\newcommand{\PARAMS}{\renewcommand{\algorithmicrequire}{\vkKw{Параметры:}}\REQUIRE}
\newcommand{\INPUTOUTPUT}{\renewcommand{\algorithmicrequire}{\vkKw{Вход и Выход:}}\REQUIRE}
\newcommand{\END}{\vskip 1ex\hrule\vskip 1ex}
\newcommand{\vkReturn}{\vkKw{вернуть} }
\newcommand{\vkExit}{\vkKw{выход}}
\newcommand{\RET}{\STATE\vkReturn}
\newcommand{\EXIT}{\STATE\vkExit}
\newcommand{\CONTINUE}{\STATE\vkKw{следующий} }
\newcommand{\IFTHEN}[1]{\STATE\algorithmicif\ #1 {\algorithmicthen}}
\newcommand{\vkProcedure}[1]{{\tt #1}}
\newcommand{\vkProc}[1]{\text{\tt #1}}
\newcommand{\PROCEDURE}[1]{\medskip\STATE\vkKw{ПРОЦЕДУРА} \vkProcedure{#1}}
\floatname{algorithm}{Алгоритм}
\floatplacement{algorithm}{!t}
% чтобы поставить точечку после номера алгоритма в \caption:
\renewcommand\floatc@ruled[2]{\vskip2pt{\@fs@cfont #1.} #2\par}
% чтобы можно было ссылаться на шаги алгоритма
\newenvironment{Algorithm}[1][t]%
{\begin{algorithm}[#1]\begin{algorithmic}[1]%
			\renewcommand{\ALC@it}{%
				\refstepcounter{ALC@line}% удивительно, почему это не сделал Peter Williams?
				\addtocounter{ALC@rem}{1}%
				\ifthenelse{\equal{\arabic{ALC@rem}}{1}}{\setcounter{ALC@rem}{0}}{}%
				\item}}%
		{\end{algorithmic}\end{algorithm}}

% Для включения графиков пакетом graphicx
\DeclareGraphicsRule{.wmf}{bmp}{}{}
\DeclareGraphicsRule{.emf}{bmp}{}{}
\DeclareGraphicsRule{.bmp}{bmp}{}{}

% Для подписей на рисунках, вставляемых includegraphics
\def\XYtext(#1,#2)#3{\rlap{\kern#1\lower-#2\hbox{#3}}}
\newcommand\XYT[3]{\rlap{\kern#1\lower-#2\hbox{#3}}}
\newcommand\XYCT[4]{\rlap{\kern#1\lower-#2\hbox{\textcolor[rgb]{#3}{#4}}}}

% Для оформления плавающих иллюстраций пакетами floatflt, caption2
\renewcommand{\captionlabeldelim}{.}

% Рисование нейронных сетей и диаграмм
\newenvironment{network}%
{\catcode`"=12\begin{xy}<1ex,0ex>:}%
	{\end{xy}\catcode`"=13}
\def\nnNode"#1"(#2)#3{\POS(#2)*#3="#1"}
\def\nnLink"#1,#2"#3{\POS"#1"\ar #3 "#2"}
\def\nnSig{%
	\underline{{}^\sigma\:\mathstrut}\vrule%
	\overline{\phantom{()}}}
\def\nnTheta{%
	\underline{{}^\theta\:\mathstrut}\vrule%
	\overline{\phantom{()}}}
\def\nnFActivation{%
	\underline{{}^\sigma\:\mathstrut}\vrule%
	\overline{\phantom{()}}}

% Скромная нумерация
\newcounter{vkItem}
\renewcommand{\thevkItem}{\arabic{vkItem}}
\newenvironment{vkItemize}{\setcounter{vkItem}{0}}{}
\newcommand{\vkItem}{\par\refstepcounter{vkItem}\thevkItem.\enspace}%  \hspace{\me}

% Оформление перечней
\renewcommand{\@listi}{
	% вертикальные промежутки:
	\topsep=0pt % вокруг списка
	\parsep=0pt % между абзацами
	\itemsep=0pt % между пунктами
	% горизонтальные промежутки:
	\itemindent=0pt % абзацный выступ
	\labelsep=1ex % расстояние до метки
	\leftmargin=\parindent % отступ слева
	\rightmargin=0pt} % отступ справа

% Перенос знака операции на следующую строку
\newcommand\brop[1]{#1\discretionary{}{\hbox{$\mathsurround=0pt #1$}}{}}

\begingroup
\catcode`\+\active\gdef+{\mathchar8235\nobreak\discretionary{}%
	{\usefont{OT1}{cmr}{m}{n}\char43}{}}
\catcode`\-\active\gdef-{\mathchar8704\nobreak\discretionary{}%
	{\usefont{OMS}{cmsy}{m}{n}\char0}{}}
\catcode`\=\active\gdef={\mathchar12349\nobreak\discretionary{}%
	{\usefont{OT1}{cmr}{m}{n}\char61}{}}
\endgroup
\def\cdot{\mathchar8705\nobreak\discretionary{}%
	{\usefont{OMS}{cmsy}{m}{n}\char1}{}}
\def\times{\mathchar8706\nobreak\discretionary{}%
	{\usefont{OMS}{cmsy}{m}{n}\char2}{}}
\AtBeginDocument{%
	\mathcode`\==32768
	\mathcode`\+=32768
	\mathcode`\-=32768
}

% На всякий случай, если пакет AmsBreakOps не подлючён
%\providecommand\brop\relax

% added by Evgeny Sokolov
\def\XX{\mathbb{X}}
\def\PP{\mathbb{P}}
\def\FF{\mathcal{F}}
\def\EE{\mathbb{E}}
\def\NN{\mathcal{N}}
\def\LL{\mathcal{L}}
\def\YY{\mathbb{Y}}
\def\AA{\mathcal{A}}
\newenvironment{esSolution}%
{\begingroup\par\noindent{\bf Решение.}}%
{\par\hfill$\scriptstyle\blacksquare$\medskip\endgroup}

\newcommand{\mat}[1]{\left(\begin{smallmatrix}#1\end{smallmatrix}\right)}

\newcommand{\KL}[2]{
	\text{KL}
	\left(\left.
	{#1}
	\mspace{3mu} \right\| \mspace{3mu}
	{#2}
	\right)
}